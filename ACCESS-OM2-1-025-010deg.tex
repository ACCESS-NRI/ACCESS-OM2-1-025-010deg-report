%%%%%%%%%%%%%%%%%%%%%%%%%%%%%%%%
%
% This will not run without gitinfo2 git hooks set up
% see https://ctan.org/pkg/gitinfo2
% This should work automatically if this file is obtained via
% git clone -- TODO: check and finish!
%
%%%%%%%%%%%%%%%%%%%%%%%%%%%%%%%%

\documentclass[11pt]{article}
\usepackage{geometry}                % See geometry.pdf to learn the layout options. There are lots.
\geometry{a4paper}                   % ... or a4paper or a5paper or ... 
%\geometry{landscape}                % Activate for for rotated page geometry
%\usepackage[parfill]{parskip}    % Activate to begin paragraphs with an empty line rather than an indent
\usepackage{graphicx}
\usepackage{amssymb}
%\usepackage{epstopdf}
%\DeclareGraphicsRule{.tif}{png}{.png}{`convert #1 `dirname #1`/`basename #1 .tif`.png}

\usepackage{ifthen}
\usepackage{color}
\definecolor{link}{rgb}{0,0,1}
\usepackage[a4paper,colorlinks,
linkcolor={link},citecolor={link},pagecolor={link},urlcolor={link},
 breaklinks, bookmarks, bookmarksopen, bookmarksnumbered
]{hyperref}
\usepackage{url}\urlstyle{sf} % rm, sf, tt or same
\usepackage{datetime2}
\usepackage[grumpy]{gitinfo2}
\DTMsetdatestyle{iso}
\usepackage{natbib}
\usepackage{longtable}
\usepackage{sistyle}
\usepackage{array}

\newcommand{\note}[1]{#1} % show all notes
%\newcommand{\note}[1]{\quad}     % hide all notes
\newcommand{\TODO}[1]{\note{\textcolor{blue}{\textsf{\textbf{TODO: #1}}}}}
\newcommand{\FIXME}[1]{\note{\textcolor{red}{\textsf{\textbf{FIXME: #1}}}}}

\title{ACCESS-OM2: The Australian Community Climate and Earth-System Simulator Ocean Model 2}
\author{Andrew Kiss, Andy Hogg, Paul Spence, Fanghua Wu, Kial Stewart,\\ Stephen Griffies, Nicholas Hannah, Russell Fiedler, Aidan Heerdegen,\\
Matthew England, Adele Morrison\\
\TODO{add anyone who's missing (order is arbitrary at this stage)}}
\date{\textsf{The latest version of this document is available from:\TODO{add link to github!}\\
\hfill{\footnotesize This version: typeset \today\ \DTMcurrenttime\ \DTMcurrentzone \\ 
\hfill Last commit%
\ifthenelse{\equal{\gitDirty}{}}{:}{ (\emph{didn't commit all tracked changes}):}
git hash: \gitAbbrevHash\ 
\gitCommitterIsoDate, \\\hfill committed to branch ``\gitBranch '' by \gitCommitterName\\
\ifthenelse{\equal{\gitRoff}{}}{}{\hfill \gitRoff\ commit(s) since release \gitRel \\} 
%\gitDirty\ 
%committed by \gitCommitterName , \gitCommitterIsoDate\ \\
\hfill\textbf{NB: git hash does not reflect any uncommitted changes to this document.}
\TODO{automatically warn if there are uncommitted changes - eg by \url{https://www.ctan.org/pkg/latexgit}}
\FIXME{is there any way include the pdf in the git repo and also have it show an up-to-date git hash?? --- see p12 of gitinfo2 documentation}}}\\
\raggedright{\vspace{10ex}
\TODO{use overleaf (with github integration? - \url{https://www.overleaf.com/blog/195-new-collaborate-online-and-offline-with-overleaf-and-git-beta}) rather than just git? - simpler for contributors}
\\
CONTRIBUTORS PLEASE NOTE:\\
\begin{itemize}
\item to make git diffs easier, please write each sentence on a separate line
\item add ``to do'' items using $\backslash$TODO\{\ldots\}
\item note errors and problems using $\backslash$FIXME\{\ldots\}
\item use a bare number (no leading v) if you do git tags (for compatibility with the gitinfo2 package used here)
\end{itemize}
}}

\begin{document}
\maketitle

\tableofcontents
\listoffigures


\section{Introduction}
This technical report documents the ACCESS-OM2 ocean-sea ice model at nominal horizontal resolutions of $1^\circ$, $0.25^\circ$ and $0.1^\circ$.

\section{Model Configuration}

\subsection{Overview}
MOM, CICE, OASIS, JRA55

cf.\ CORE \citep{GriffiesBiastochBoningBryanDanabasogluChassignetEnglandGerdesHaak2009a}, CORE-II \citep{DanabasogluYeagerBaileyBehrensBentsenBiBiastochBoningBozec2014a}

\subsection{MOM version and settings}
\TODO{auto-generate namelist table}

\subsection{Grid}
Discuss KDS vertical grid \citet{StewartHoggGriffiesHeerdegenWardSpenceEngland2017a}


\subsection{Bathymetry}
based on Gebco2014 30sec gridded data % located at /g/data3/hh5/tmp/cosima/bathymetry/gebco.nc
which exact version? \url{http://www.gebco.net/data_and_products/gridded_bathymetry_data/gebco_30_second_grid/}

\subsection{CICE}

\subsection{OASIS}

Nic's work on ESMF regridding


\subsection{Forcing}

JRA55 version

discuss choice of year for RYF

Regridding method

Runoff - incl distributed iceberg melt?

\subsection{Initial conditions and spinup}

\section{Model computational performance}

\section{Model evaluation}
 - do this via a cookbook that emits all needed pngs/pdfs 

\subsection{Global conservation and drift}

\subsubsection{Heat}

\subsubsection{Salt}

\subsubsection{Fresh water}

\subsubsection{Sea ice}


\subsection{Transports through key straits and boundary currents}

\subsection{Overturning}

\subsection{Water mass properties}

\subsection{EKE}

\subsection{Time-dependence, eg of WBCs}

\subsection{Sea ice - area, thickness, formation rate}

\appendix
\section{Auto-generated namelists}

\definecolor{hilite}{rgb}{1,1,0}
%\newcommand{\differ}[1]{#1} % plain display
%\newcommand{\differ}[1]{\textbf{#1}} % bold display
\newcommand{\differ}[1]{\colorbox{hilite}{#1}} % colour highlight
\newcommand{\link}[2]{#1} % plain display
\setlength{\fboxsep}{0pt}

These are auto-generated by make\_nml\_tables.sh.
Variables are \textcolor{link}{weblinks} to source code searches and those that differ between the models are \differ{\textcolor{link}{highlighted}}.

\subsection{MOM namelist `input.nml'}
\renewcommand{\link}[2]{\href{https://github.com/mom-ocean/MOM5/search?q=#2}{#1}} % link to documentation (requires hyperref package)
{\tiny \input{ocean_input_nml.tex}}

\subsection{CICE namelists}
\renewcommand{\link}[2]{\href{https://github.com/OceansAus/cice5/search?q=#2}{#1}} % link to documentation (requires hyperref package)
\TODO{work out which ice nmls are relevant}
\subsubsection{input\_ice.nml}
{\tiny 
% Latex tabulation of Fortran namelist, auto-generated by nmltab.py <https://github.com/aekiss/nmltab>
%
% Include this file in a latex document using \import{path/to/this/file}.
% The importing document requires
% \usepackage{ltablex, array, sistyle}
% and possibly (depending on definitions below)
% \usepackage{hyperref, color}
% and also needs to define 'nmldiffer', 'nmllink' and 'ignored' commands, e.g.
% \newcommand{\nmldiffer}[1]{#1} % no special display of differing variables
% \newcommand{\nmldiffer}[1]{\textbf{#1}} % bold display of differing variables
% \definecolor{hilite}{cmyk}{0, 0, 0.9, 0}\newcommand{\nmldiffer}[1]{\colorbox{hilite}{#1}}\setlength{\fboxsep}{0pt} % colour highlight of differing variables (requires color package)
% \newcommand{\nmllink}[2]{#1} % don't link variables
% \newcommand{\nmllink}[2]{\href{https://github.com/mom-ocean/MOM5/search?q=#2}{#1}} % link variables to documentation (requires hyperref package)
% \newcommand{\ignored}[1]{#1} % no special display of ignored variables
% \definecolor{ignore}{gray}{0.7}\newcommand{\ignored}[1]{\textcolor{ignore}{#1}} % gray display of ignored variables (but only in groups where masterswitch key is present and false, so may not work well for differences; requires color package)
% and also define the length 'nmllen' that sets the column width, e.g.
% \newlength{\nmllen}\setlength{\nmllen}{12ex}

\newcolumntype{R}{>{\raggedleft\arraybackslash}p{\nmllen}}
\begin{tabularx}{\linewidth}{XRRR}
\hline
\textbf{Group\quad\hfill Variable}	 & 	\textbf{.\slash raijin\slash g\slash data3\slash hh5\slash tmp\slash cosima\slash access-om2\slash 1deg\_jra55v13\_iaf\_spinup1\_B1\_lastcycle\slash output059\slash ice\slash input\_ice.nml}	 & 	\textbf{.\slash raijin\slash g\slash data3\slash hh5\slash tmp\slash cosima\slash access-om2-025\slash 025deg\_jra55v13\_iaf\_gmredi6\slash output117\slash ice\slash input\_ice.nml}	 & 	\textbf{.\slash raijin\slash g\slash data3\slash hh5\slash tmp\slash cosima\slash access-om2-01\slash 01deg\_jra55v13\_iaf\slash output197\slash ice\slash input\_ice.nml} \\
\hline\endfirsthead
\hline
\textbf{Group (continued)\quad\hfill Variable}	 & 	\textbf{.\slash raijin\slash g\slash data3\slash hh5\slash tmp\slash cosima\slash access-om2\slash 1deg\_jra55v13\_iaf\_spinup1\_B1\_lastcycle\slash output059\slash ice\slash input\_ice.nml}	 & 	\textbf{.\slash raijin\slash g\slash data3\slash hh5\slash tmp\slash cosima\slash access-om2-025\slash 025deg\_jra55v13\_iaf\_gmredi6\slash output117\slash ice\slash input\_ice.nml}	 & 	\textbf{.\slash raijin\slash g\slash data3\slash hh5\slash tmp\slash cosima\slash access-om2-01\slash 01deg\_jra55v13\_iaf\slash output197\slash ice\slash input\_ice.nml} \\
\hline\endhead
\&\nmllink{coupling\_nml}{coupling_nml} \hfill \nmllink{chk\_a2i\_fields}{chk_a2i_fields}	 & 	False	 & 	False	 & 	False \\
 \hfill \nmldiffer{\nmllink{chk\_frzmlt\_sst}{chk_frzmlt_sst}}	 & 		 & 	False	 & 	False \\
 \hfill \nmllink{chk\_gfdl\_roughness}{chk_gfdl_roughness}	 & 	False	 & 	False	 & 	False \\
 \hfill \nmldiffer{\nmllink{chk\_i2a\_fields}{chk_i2a_fields}}	 & 		 & 	False	 & 	False \\
 \hfill \nmldiffer{\nmllink{chk\_i2o\_fields}{chk_i2o_fields}}	 & 		 & 	False	 & 	False \\
 \hfill \nmldiffer{\nmllink{chk\_o2i\_fields}{chk_o2i_fields}}	 & 		 & 	False	 & 	False \\
 \hfill \nmllink{cst\_ocn\_albedo}{cst_ocn_albedo}	 & 	True	 & 	True	 & 	True \\
 \hfill \nmllink{gfdl\_surface\_flux}{gfdl_surface_flux}	 & 	True	 & 	True	 & 	True \\
 \hfill \nmllink{ice\_fwflux}{ice_fwflux}	 & 	True	 & 	True	 & 	True \\
 \hfill \nmllink{ice\_pressure\_on}{ice_pressure_on}	 & 	True	 & 	True	 & 	True \\
 \hfill \nmllink{limit\_icemelt}{limit_icemelt}	 & 	False	 & 	False	 & 	False \\
 \hfill \nmllink{meltlimit}{meltlimit}	 & 	\num*{-200.0}{}	 & 	\num*{-200.0}{}	 & 	\num*{-200.0}{} \\
 \hfill \nmllink{ocn\_albedo}{ocn_albedo}	 & 	\num*{0.1}{}	 & 	\num*{0.1}{}	 & 	\num*{0.1}{} \\
 \hfill \nmllink{pop\_icediag}{pop_icediag}	 & 	True	 & 	True	 & 	True \\
 \hfill \nmllink{precip\_factor}{precip_factor}	 & 	\num*{1.0}{}	 & 	\num*{1.0}{}	 & 	\num*{1.0}{} \\
 \hfill \nmllink{rotate\_winds}{rotate_winds}	 & 	True	 & 	True	 & 	True \\
 \hfill \nmllink{use\_ocnslope}{use_ocnslope}	 & 	False	 & 	False	 & 	False \\
 \hfill \nmllink{use\_umask}{use_umask}	 & 	False	 & 	False	 & 	False \\
\hline
\end{tabularx}
}
\subsubsection{input\_ice\_gfdl.nml}
{\tiny 
% Latex tabulation of Fortran namelist, auto-generated by nmltab.py <https://github.com/aekiss/nmltab>
%
% Include this file in a latex document using \import{path/to/this/file}.
% The importing document requires
% \usepackage{ltablex, array, sistyle}
% and possibly (depending on definitions below)
% \usepackage{hyperref, color}
% and also needs to define 'nmldiffer', 'nmllink' and 'ignored' commands, e.g.
% \newcommand{\nmldiffer}[1]{#1} % no special display of differing variables
% \newcommand{\nmldiffer}[1]{\textbf{#1}} % bold display of differing variables
% \definecolor{hilite}{cmyk}{0, 0, 0.9, 0}\newcommand{\nmldiffer}[1]{\colorbox{hilite}{#1}}\setlength{\fboxsep}{0pt} % colour highlight of differing variables (requires color package)
% \newcommand{\nmllink}[2]{#1} % don't link variables
% \newcommand{\nmllink}[2]{\href{https://github.com/mom-ocean/MOM5/search?q=#2}{#1}} % link variables to documentation (requires hyperref package)
% \newcommand{\ignored}[1]{#1} % no special display of ignored variables
% \definecolor{ignore}{gray}{0.7}\newcommand{\ignored}[1]{\textcolor{ignore}{#1}} % gray display of ignored variables (but only in groups where masterswitch key is present and false, so may not work well for differences; requires color package)
% and also define the length 'nmllen' that sets the column width, e.g.
% \newlength{\nmllen}\setlength{\nmllen}{12ex}

\newcolumntype{R}{>{\raggedleft\arraybackslash}p{\nmllen}}
\begin{tabularx}{\linewidth}{XRRRR}
\hline
\textbf{Group\quad\hfill Variable}	 & 	\textbf{.\slash raijin\slash g\slash data3\slash hh5\slash tmp\slash cosima\slash access-om2\slash 1deg\_jra55v13\_iaf\_spinup1\_A\slash output059\slash ice\slash input\_ice\_gfdl.nml}	 & 	\textbf{.\slash raijin\slash g\slash data3\slash hh5\slash tmp\slash cosima\slash access-om2-025\slash 025deg\_jra55v13\_iaf\slash output097\slash ice\slash input\_ice\_gfdl.nml}	 & 	\textbf{.\slash raijin\slash g\slash data3\slash hh5\slash tmp\slash cosima\slash access-om2-025\slash 025deg\_jra55v13\_ryf8485\_spinup\_A\slash output099\slash ice\slash input\_ice\_gfdl.nml}	 & 	\textbf{.\slash raijin\slash g\slash data3\slash hh5\slash tmp\slash cosima\slash access-om2-01\slash 01deg\_jra55v13\_ryf8485\_spinup6\slash output421\slash ice\slash input\_ice\_gfdl.nml} \\
\hline\endfirsthead
\hline
\textbf{Group (continued)\quad\hfill Variable}	 & 	\textbf{.\slash raijin\slash g\slash data3\slash hh5\slash tmp\slash cosima\slash access-om2\slash 1deg\_jra55v13\_iaf\_spinup1\_A\slash output059\slash ice\slash input\_ice\_gfdl.nml}	 & 	\textbf{.\slash raijin\slash g\slash data3\slash hh5\slash tmp\slash cosima\slash access-om2-025\slash 025deg\_jra55v13\_iaf\slash output097\slash ice\slash input\_ice\_gfdl.nml}	 & 	\textbf{.\slash raijin\slash g\slash data3\slash hh5\slash tmp\slash cosima\slash access-om2-025\slash 025deg\_jra55v13\_ryf8485\_spinup\_A\slash output099\slash ice\slash input\_ice\_gfdl.nml}	 & 	\textbf{.\slash raijin\slash g\slash data3\slash hh5\slash tmp\slash cosima\slash access-om2-01\slash 01deg\_jra55v13\_ryf8485\_spinup6\slash output421\slash ice\slash input\_ice\_gfdl.nml} \\
\hline\endhead
\&\nmllink{ocean\_rough\_nml}{ocean_rough_nml} \hfill \nmllink{charnock}{charnock}	 & 	\num*{0.032}{}	 & 	\num*{0.032}{}	 & 	\num*{0.032}{}	 & 	\num*{0.032}{} \\
 \hfill \nmllink{do\_cap40}{do_cap40}	 & 	False	 & 	False	 & 	False	 & 	False \\
 \hfill \nmllink{do\_highwind}{do_highwind}	 & 	False	 & 	False	 & 	False	 & 	False \\
 \hfill \nmllink{rough\_scheme}{rough_scheme}	 & 	'beljaars'	 & 	'beljaars'	 & 	'beljaars'	 & 	'beljaars' \\
 \hfill \nmllink{roughness\_heat}{roughness_heat}	 & 	\num*{5.8e-5}{}	 & 	\num*{5.8e-5}{}	 & 	\num*{5.8e-5}{}	 & 	\num*{5.8e-5}{} \\
 \hfill \nmllink{roughness\_min}{roughness_min}	 & 	\num*{1e-6}{}	 & 	\num*{1e-6}{}	 & 	\num*{1e-6}{}	 & 	\num*{1e-6}{} \\
 \hfill \nmllink{roughness\_moist}{roughness_moist}	 & 	\num*{5.8e-5}{}	 & 	\num*{5.8e-5}{}	 & 	\num*{5.8e-5}{}	 & 	\num*{5.8e-5}{} \\
 \hfill \nmllink{roughness\_mom}{roughness_mom}	 & 	\num*{5.8e-5}{}	 & 	\num*{5.8e-5}{}	 & 	\num*{5.8e-5}{}	 & 	\num*{5.8e-5}{} \\
 \hfill \nmllink{zcoh1}{zcoh1}	 & 	\num*{0.0}{}	 & 	\num*{0.0}{}	 & 	\num*{0.0}{}	 & 	\num*{0.0}{} \\
 \hfill \nmllink{zcoq1}{zcoq1}	 & 	\num*{0.0}{}	 & 	\num*{0.0}{}	 & 	\num*{0.0}{}	 & 	\num*{0.0}{} \\
\hline
\&\nmllink{surface\_flux\_nml}{surface_flux_nml} \hfill \nmllink{alt\_gustiness}{alt_gustiness}	 & 	False	 & 	False	 & 	False	 & 	False \\
 \hfill \nmllink{gust\_const}{gust_const}	 & 	\num*{1.0}{}	 & 	\num*{1.0}{}	 & 	\num*{1.0}{}	 & 	\num*{1.0}{} \\
 \hfill \nmllink{gust\_min}{gust_min}	 & 	\num*{0.0}{}	 & 	\num*{0.0}{}	 & 	\num*{0.0}{}	 & 	\num*{0.0}{} \\
 \hfill \nmllink{ncar\_ocean\_flux}{ncar_ocean_flux}	 & 	True	 & 	True	 & 	True	 & 	True \\
 \hfill \nmllink{ncar\_ocean\_flux\_orig}{ncar_ocean_flux_orig}	 & 	False	 & 	False	 & 	False	 & 	False \\
 \hfill \nmllink{no\_neg\_q}{no_neg_q}	 & 	False	 & 	False	 & 	False	 & 	False \\
 \hfill \nmllink{old\_dtaudv}{old_dtaudv}	 & 	False	 & 	False	 & 	False	 & 	False \\
 \hfill \nmllink{raoult\_sat\_vap}{raoult_sat_vap}	 & 	False	 & 	False	 & 	False	 & 	False \\
 \hfill \nmllink{use\_mixing\_ratio}{use_mixing_ratio}	 & 	False	 & 	False	 & 	False	 & 	False \\
 \hfill \nmllink{use\_virtual\_temp}{use_virtual_temp}	 & 	True	 & 	True	 & 	True	 & 	True \\
\hline
\end{tabularx}
}
\subsubsection{input\_ice\_monin.nml}
{\tiny 
% Latex tabulation of Fortran namelist, auto-generated by nmltab.py <https://github.com/aekiss/nmltab>
%
% Include this file in a latex document using \import{path/to/this/file}.
% The importing document requires
% \usepackage{ltablex, array, sistyle}
% and possibly (depending on definitions below)
% \usepackage{hyperref, color}
% and also needs to define 'nmldiffer', 'nmllink' and 'ignored' commands, e.g.
% \newcommand{\nmldiffer}[1]{#1} % no special display of differing variables
% \newcommand{\nmldiffer}[1]{\textbf{#1}} % bold display of differing variables
% \definecolor{hilite}{cmyk}{0, 0, 0.9, 0}\newcommand{\nmldiffer}[1]{\colorbox{hilite}{#1}}\setlength{\fboxsep}{0pt} % colour highlight of differing variables (requires color package)
% \newcommand{\nmllink}[2]{#1} % don't link variables
% \newcommand{\nmllink}[2]{\href{https://github.com/mom-ocean/MOM5/search?q=#2}{#1}} % link variables to documentation (requires hyperref package)
% \newcommand{\ignored}[1]{#1} % no special display of ignored variables
% \definecolor{ignore}{gray}{0.7}\newcommand{\ignored}[1]{\textcolor{ignore}{#1}} % gray display of ignored variables (but only in groups where masterswitch key is present and false, so may not work well for differences; requires color package)
% and also define the length 'nmllen' that sets the column width, e.g.
% \newlength{\nmllen}\setlength{\nmllen}{12ex}

\newcolumntype{R}{>{\raggedleft\arraybackslash}p{\nmllen}}
\begin{tabularx}{\linewidth}{XRRR}
\hline
\textbf{Group\quad\hfill Variable}	 & 	\textbf{.\slash raijin\slash g\slash data3\slash hh5\slash tmp\slash cosima\slash access-om2\slash 1deg\_jra55v13\_iaf\_spinup1\_A\slash output059\slash ice\slash input\_ice\_monin.nml}	 & 	\textbf{.\slash raijin\slash g\slash data3\slash hh5\slash tmp\slash cosima\slash access-om2-025\slash 025deg\_jra55v13\_iaf\_gmredi\slash output149\slash ice\slash input\_ice\_monin.nml}	 & 	\textbf{.\slash raijin\slash g\slash data3\slash hh5\slash tmp\slash cosima\slash access-om2-01\slash 01deg\_jra55v13\_iaf\slash output191\slash ice\slash input\_ice\_monin.nml} \\
\hline\endfirsthead
\hline
\textbf{Group (continued)\quad\hfill Variable}	 & 	\textbf{.\slash raijin\slash g\slash data3\slash hh5\slash tmp\slash cosima\slash access-om2\slash 1deg\_jra55v13\_iaf\_spinup1\_A\slash output059\slash ice\slash input\_ice\_monin.nml}	 & 	\textbf{.\slash raijin\slash g\slash data3\slash hh5\slash tmp\slash cosima\slash access-om2-025\slash 025deg\_jra55v13\_iaf\_gmredi\slash output149\slash ice\slash input\_ice\_monin.nml}	 & 	\textbf{.\slash raijin\slash g\slash data3\slash hh5\slash tmp\slash cosima\slash access-om2-01\slash 01deg\_jra55v13\_iaf\slash output191\slash ice\slash input\_ice\_monin.nml} \\
\hline\endhead
\&\nmllink{monin\_obukhov\_nml}{monin_obukhov_nml} \hfill \nmllink{neutral}{neutral}	 & 	True	 & 	True	 & 	True \\
\hline
\end{tabularx}
}
\subsubsection{cice\_in.nml}
{\tiny 
% Latex tabulation of Fortran namelist, auto-generated by nmltab.py <https://github.com/aekiss/nmltab>
%
% Include this file in a latex document using \import{path/to/this/file}.
% The importing document requires
% \usepackage{ltablex, array, sistyle}
% and possibly (depending on definitions below)
% \usepackage{hyperref, color}
% and also needs to define 'nmldiffer', 'nmllink' and 'ignored' commands, e.g.
% \newcommand{\nmldiffer}[1]{#1} % no special display of differing variables
% \newcommand{\nmldiffer}[1]{\textbf{#1}} % bold display of differing variables
% \definecolor{hilite}{cmyk}{0, 0, 0.9, 0}\newcommand{\nmldiffer}[1]{\colorbox{hilite}{#1}}\setlength{\fboxsep}{0pt} % colour highlight of differing variables (requires color package)
% \newcommand{\nmllink}[2]{#1} % don't link variables
% \newcommand{\nmllink}[2]{\href{https://github.com/mom-ocean/MOM5/search?q=#2}{#1}} % link variables to documentation (requires hyperref package)
% \newcommand{\ignored}[1]{#1} % no special display of ignored variables
% \definecolor{ignore}{gray}{0.7}\newcommand{\ignored}[1]{\textcolor{ignore}{#1}} % gray display of ignored variables (but only in groups where masterswitch key is present and false, so may not work well for differences; requires color package)
% and also define the length 'nmllen' that sets the column width, e.g.
% \newlength{\nmllen}\setlength{\nmllen}{12ex}

\newcolumntype{R}{>{\raggedleft\arraybackslash}p{\nmllen}}
\begin{tabularx}{\linewidth}{XRRRR}
\hline
\textbf{Group\quad\hfill Variable}	 & 	\textbf{.\slash raijin\slash g\slash data3\slash hh5\slash tmp\slash cosima\slash access-om2\slash 1deg\_jra55v13\_iaf\_spinup1\_A\slash output059\slash ice\slash cice\_in.nml}	 & 	\textbf{.\slash raijin\slash g\slash data3\slash hh5\slash tmp\slash cosima\slash access-om2-025\slash 025deg\_jra55v13\_iaf\slash output148\slash ice\slash cice\_in.nml}	 & 	\textbf{.\slash raijin\slash g\slash data3\slash hh5\slash tmp\slash cosima\slash access-om2-025\slash 025deg\_jra55v13\_ryf8485\_KDS50\slash output075\slash ice\slash cice\_in.nml}	 & 	\textbf{.\slash raijin\slash g\slash data3\slash hh5\slash tmp\slash cosima\slash access-om2-01\slash 01deg\_jra55v13\_ryf8485\_spinup6\slash output423\slash ice\slash cice\_in.nml} \\
\hline\endfirsthead
\hline
\textbf{Group (continued)\quad\hfill Variable}	 & 	\textbf{.\slash raijin\slash g\slash data3\slash hh5\slash tmp\slash cosima\slash access-om2\slash 1deg\_jra55v13\_iaf\_spinup1\_A\slash output059\slash ice\slash cice\_in.nml}	 & 	\textbf{.\slash raijin\slash g\slash data3\slash hh5\slash tmp\slash cosima\slash access-om2-025\slash 025deg\_jra55v13\_iaf\slash output148\slash ice\slash cice\_in.nml}	 & 	\textbf{.\slash raijin\slash g\slash data3\slash hh5\slash tmp\slash cosima\slash access-om2-025\slash 025deg\_jra55v13\_ryf8485\_KDS50\slash output075\slash ice\slash cice\_in.nml}	 & 	\textbf{.\slash raijin\slash g\slash data3\slash hh5\slash tmp\slash cosima\slash access-om2-01\slash 01deg\_jra55v13\_ryf8485\_spinup6\slash output423\slash ice\slash cice\_in.nml} \\
\hline\endhead
\&\nmllink{domain\_nml}{domain_nml} \hfill \nmldiffer{\nmllink{distribution\_type}{distribution_type}}	 & 	'cartesian'	 & 	'roundrobin'	 & 	'roundrobin'	 & 	'cartesian' \\
 \hfill \nmllink{distribution\_wght}{distribution_wght}	 & 	'latitude'	 & 	'latitude'	 & 	'latitude'	 & 	'latitude' \\
 \hfill \nmllink{ew\_boundary\_type}{ew_boundary_type}	 & 	'cyclic'	 & 	'cyclic'	 & 	'cyclic'	 & 	'cyclic' \\
 \hfill \nmllink{maskhalo\_bound}{maskhalo_bound}	 & 	True	 & 	True	 & 	True	 & 	True \\
 \hfill \nmllink{maskhalo\_dyn}{maskhalo_dyn}	 & 	True	 & 	True	 & 	True	 & 	True \\
 \hfill \nmllink{maskhalo\_remap}{maskhalo_remap}	 & 	True	 & 	True	 & 	True	 & 	True \\
 \hfill \nmldiffer{\nmllink{nprocs}{nprocs}}	 & 	24	 & 	393	 & 	393	 & 	2000 \\
 \hfill \nmllink{ns\_boundary\_type}{ns_boundary_type}	 & 	'tripole'	 & 	'tripole'	 & 	'tripole'	 & 	'tripole' \\
 \hfill \nmldiffer{\nmllink{processor\_shape}{processor_shape}}	 & 	'slenderX1'	 & 	'square-ice'	 & 	'square-ice'	 & 	'square-ice' \\
\hline
\&\nmllink{dynamics\_nml}{dynamics_nml} \hfill \nmllink{advection}{advection}	 & 	'remap'	 & 	'remap'	 & 	'remap'	 & 	'remap' \\
 \hfill \nmldiffer{\nmllink{cosw}{cosw}}	 & 	\num*{0.96}{}	 & 	\num*{1.0}{}	 & 	\num*{1.0}{}	 & 	\num*{1.0}{} \\
 \hfill \nmllink{dragio}{dragio}	 & 	\num*{0.00536}{}	 & 	\num*{0.00536}{}	 & 	\num*{0.00536}{}	 & 	\num*{0.00536}{} \\
 \hfill \nmllink{iceruf}{iceruf}	 & 	\num*{0.0005}{}	 & 	\num*{0.0005}{}	 & 	\num*{0.0005}{}	 & 	\num*{0.0005}{} \\
 \hfill \nmllink{kdyn}{kdyn}	 & 	1	 & 	1	 & 	1	 & 	1 \\
 \hfill \nmllink{krdg\_partic}{krdg_partic}	 & 	1	 & 	1	 & 	1	 & 	1 \\
 \hfill \nmllink{krdg\_redist}{krdg_redist}	 & 	1	 & 	1	 & 	1	 & 	1 \\
 \hfill \nmllink{kstrength}{kstrength}	 & 	1	 & 	1	 & 	1	 & 	1 \\
 \hfill \nmllink{mu\_rdg}{mu_rdg}	 & 	3	 & 	3	 & 	3	 & 	3 \\
 \hfill \nmllink{ndte}{ndte}	 & 	120	 & 	120	 & 	120	 & 	120 \\
 \hfill \nmllink{revised\_evp}{revised_evp}	 & 	False	 & 	False	 & 	False	 & 	False \\
 \hfill \nmldiffer{\nmllink{sinw}{sinw}}	 & 	\num*{0.28}{}	 & 	\num*{0.0}{}	 & 	\num*{0.0}{}	 & 	\num*{0.0}{} \\
\hline
\&\nmllink{forcing\_nml}{forcing_nml} \hfill \nmllink{atm\_data\_dir}{atm_data_dir}	 & 	'unknown\_atm\_data\_dir'	 & 	'unknown\_atm\_data\_dir'	 & 	'unknown\_atm\_data\_dir'	 & 	'unknown\_atm\_data\_dir' \\
 \hfill \nmllink{atm\_data\_format}{atm_data_format}	 & 	'nc'	 & 	'nc'	 & 	'nc'	 & 	'nc' \\
 \hfill \nmllink{atm\_data\_type}{atm_data_type}	 & 	'default'	 & 	'default'	 & 	'default'	 & 	'default' \\
 \hfill \nmllink{atmbndy}{atmbndy}	 & 	'default'	 & 	'default'	 & 	'default'	 & 	'default' \\
 \hfill \nmllink{calc\_strair}{calc_strair}	 & 	True	 & 	True	 & 	True	 & 	True \\
 \hfill \nmllink{calc\_tsfc}{calc_tsfc}	 & 	True	 & 	True	 & 	True	 & 	True \\
 \hfill \nmllink{formdrag}{formdrag}	 & 	False	 & 	False	 & 	False	 & 	False \\
 \hfill \nmllink{fyear\_init}{fyear_init}	 & 	1	 & 	1	 & 	1	 & 	1 \\
 \hfill \nmllink{oceanmixed\_file}{oceanmixed_file}	 & 	'unknown\_oceanmixed\_file'	 & 	'unknown\_oceanmixed\_file'	 & 	'unknown\_oceanmixed\_file'	 & 	'unknown\_oceanmixed\_file' \\
 \hfill \nmllink{oceanmixed\_ice}{oceanmixed_ice}	 & 	False	 & 	False	 & 	False	 & 	False \\
 \hfill \nmllink{ocn\_data\_dir}{ocn_data_dir}	 & 	'unknown\_ocn\_data\_dir'	 & 	'unknown\_ocn\_data\_dir'	 & 	'unknown\_ocn\_data\_dir'	 & 	'unknown\_ocn\_data\_dir' \\
 \hfill \nmllink{ocn\_data\_format}{ocn_data_format}	 & 	'nc'	 & 	'nc'	 & 	'nc'	 & 	'nc' \\
 \hfill \nmllink{precip\_units}{precip_units}	 & 	'mks'	 & 	'mks'	 & 	'mks'	 & 	'mks' \\
 \hfill \nmllink{restore\_ice}{restore_ice}	 & 	False	 & 	False	 & 	False	 & 	False \\
 \hfill \nmllink{restore\_sst}{restore_sst}	 & 	False	 & 	False	 & 	False	 & 	False \\
 \hfill \nmllink{sss\_data\_type}{sss_data_type}	 & 	'default'	 & 	'default'	 & 	'default'	 & 	'default' \\
 \hfill \nmllink{sst\_data\_type}{sst_data_type}	 & 	'default'	 & 	'default'	 & 	'default'	 & 	'default' \\
 \hfill \nmldiffer{\nmllink{tfrz\_option}{tfrz_option}}	 & 	'linear\_salt'	 & 	'linear\_salt'	 & 	'linear\_salt'	 & 	'mushy' \\
 \hfill \nmllink{trestore}{trestore}	 & 	0	 & 	0	 & 	0	 & 	0 \\
 \hfill \nmllink{update\_ocn\_f}{update_ocn_f}	 & 	True	 & 	True	 & 	True	 & 	True \\
 \hfill \nmllink{ustar\_min}{ustar_min}	 & 	\num*{0.0005}{}	 & 	\num*{0.0005}{}	 & 	\num*{0.0005}{}	 & 	\num*{0.0005}{} \\
 \hfill \nmllink{ycycle}{ycycle}	 & 	1	 & 	1	 & 	1	 & 	1 \\
\hline
\&\nmllink{grid\_nml}{grid_nml} \hfill \nmllink{grid\_file}{grid_file}	 & 	'RESTART\slash grid.nc'	 & 	'RESTART\slash grid.nc'	 & 	'RESTART\slash grid.nc'	 & 	'RESTART\slash grid.nc' \\
 \hfill \nmllink{grid\_format}{grid_format}	 & 	'nc'	 & 	'nc'	 & 	'nc'	 & 	'nc' \\
 \hfill \nmllink{grid\_type}{grid_type}	 & 	'tripole'	 & 	'tripole'	 & 	'tripole'	 & 	'tripole' \\
 \hfill \nmllink{kcatbound}{kcatbound}	 & 	0	 & 	0	 & 	0	 & 	0 \\
 \hfill \nmllink{kmt\_file}{kmt_file}	 & 	'RESTART\slash kmt.nc'	 & 	'RESTART\slash kmt.nc'	 & 	'RESTART\slash kmt.nc'	 & 	'RESTART\slash kmt.nc' \\
\hline
\&\nmllink{icefields\_bgc\_nml}{icefields_bgc_nml} \hfill \nmllink{f\_aero}{f_aero}	 & 	'x'	 & 	'x'	 & 	'x'	 & 	'x' \\
 \hfill \nmllink{f\_bgc\_am\_ml}{f_bgc_am_ml}	 & 	'x'	 & 	'x'	 & 	'x'	 & 	'x' \\
 \hfill \nmllink{f\_bgc\_am\_sk}{f_bgc_am_sk}	 & 	'x'	 & 	'x'	 & 	'x'	 & 	'x' \\
 \hfill \nmllink{f\_bgc\_c\_sk}{f_bgc_c_sk}	 & 	'x'	 & 	'x'	 & 	'x'	 & 	'x' \\
 \hfill \nmllink{f\_bgc\_chl\_sk}{f_bgc_chl_sk}	 & 	'x'	 & 	'x'	 & 	'x'	 & 	'x' \\
 \hfill \nmllink{f\_bgc\_dms\_sk}{f_bgc_dms_sk}	 & 	'x'	 & 	'x'	 & 	'x'	 & 	'x' \\
 \hfill \nmllink{f\_bgc\_dmsp\_ml}{f_bgc_dmsp_ml}	 & 	'x'	 & 	'x'	 & 	'x'	 & 	'x' \\
 \hfill \nmllink{f\_bgc\_dmspd\_sk}{f_bgc_dmspd_sk}	 & 	'x'	 & 	'x'	 & 	'x'	 & 	'x' \\
 \hfill \nmllink{f\_bgc\_dmspp\_sk}{f_bgc_dmspp_sk}	 & 	'x'	 & 	'x'	 & 	'x'	 & 	'x' \\
 \hfill \nmllink{f\_bgc\_n\_sk}{f_bgc_n_sk}	 & 	'x'	 & 	'x'	 & 	'x'	 & 	'x' \\
 \hfill \nmllink{f\_bgc\_nit\_ml}{f_bgc_nit_ml}	 & 	'x'	 & 	'x'	 & 	'x'	 & 	'x' \\
 \hfill \nmllink{f\_bgc\_nit\_sk}{f_bgc_nit_sk}	 & 	'x'	 & 	'x'	 & 	'x'	 & 	'x' \\
 \hfill \nmllink{f\_bgc\_sil\_ml}{f_bgc_sil_ml}	 & 	'x'	 & 	'x'	 & 	'x'	 & 	'x' \\
 \hfill \nmllink{f\_bgc\_sil\_sk}{f_bgc_sil_sk}	 & 	'x'	 & 	'x'	 & 	'x'	 & 	'x' \\
 \hfill \nmllink{f\_bphi}{f_bphi}	 & 	'x'	 & 	'x'	 & 	'x'	 & 	'x' \\
 \hfill \nmllink{f\_btin}{f_btin}	 & 	'x'	 & 	'x'	 & 	'x'	 & 	'x' \\
 \hfill \nmllink{f\_faero\_atm}{f_faero_atm}	 & 	'x'	 & 	'x'	 & 	'x'	 & 	'x' \\
 \hfill \nmllink{f\_faero\_ocn}{f_faero_ocn}	 & 	'x'	 & 	'x'	 & 	'x'	 & 	'x' \\
 \hfill \nmldiffer{\nmllink{f\_fbri}{f_fbri}}	 & 	'm'	 & 	'm'	 & 	'm'	 & 	'x' \\
 \hfill \nmllink{f\_fn}{f_fn}	 & 	'x'	 & 	'x'	 & 	'x'	 & 	'x' \\
 \hfill \nmllink{f\_fn\_ai}{f_fn_ai}	 & 	'x'	 & 	'x'	 & 	'x'	 & 	'x' \\
 \hfill \nmllink{f\_fnh}{f_fnh}	 & 	'x'	 & 	'x'	 & 	'x'	 & 	'x' \\
 \hfill \nmllink{f\_fnh\_ai}{f_fnh_ai}	 & 	'x'	 & 	'x'	 & 	'x'	 & 	'x' \\
 \hfill \nmllink{f\_fno}{f_fno}	 & 	'x'	 & 	'x'	 & 	'x'	 & 	'x' \\
 \hfill \nmllink{f\_fno\_ai}{f_fno_ai}	 & 	'x'	 & 	'x'	 & 	'x'	 & 	'x' \\
 \hfill \nmllink{f\_fsil}{f_fsil}	 & 	'x'	 & 	'x'	 & 	'x'	 & 	'x' \\
 \hfill \nmllink{f\_fsil\_ai}{f_fsil_ai}	 & 	'x'	 & 	'x'	 & 	'x'	 & 	'x' \\
 \hfill \nmllink{f\_grownet}{f_grownet}	 & 	'x'	 & 	'x'	 & 	'x'	 & 	'x' \\
 \hfill \nmldiffer{\nmllink{f\_hbri}{f_hbri}}	 & 	'm'	 & 	'm'	 & 	'm'	 & 	'x' \\
 \hfill \nmllink{f\_ppnet}{f_ppnet}	 & 	'x'	 & 	'x'	 & 	'x'	 & 	'x' \\
\hline
\&\nmllink{icefields\_drag\_nml}{icefields_drag_nml} \hfill \nmllink{f\_cdn\_atm}{f_cdn_atm}	 & 	'x'	 & 	'x'	 & 	'x'	 & 	'x' \\
 \hfill \nmllink{f\_cdn\_ocn}{f_cdn_ocn}	 & 	'x'	 & 	'x'	 & 	'x'	 & 	'x' \\
 \hfill \nmllink{f\_drag}{f_drag}	 & 	'x'	 & 	'x'	 & 	'x'	 & 	'x' \\
\hline
\&\nmllink{icefields\_mechred\_nml}{icefields_mechred_nml} \hfill \nmllink{f\_alvl}{f_alvl}	 & 	'm'	 & 	'm'	 & 	'm'	 & 	'm' \\
 \hfill \nmllink{f\_aparticn}{f_aparticn}	 & 	'x'	 & 	'x'	 & 	'x'	 & 	'x' \\
 \hfill \nmllink{f\_araftn}{f_araftn}	 & 	'x'	 & 	'x'	 & 	'x'	 & 	'x' \\
 \hfill \nmllink{f\_ardg}{f_ardg}	 & 	'm'	 & 	'm'	 & 	'm'	 & 	'm' \\
 \hfill \nmllink{f\_ardgn}{f_ardgn}	 & 	'x'	 & 	'x'	 & 	'x'	 & 	'x' \\
 \hfill \nmllink{f\_aredistn}{f_aredistn}	 & 	'x'	 & 	'x'	 & 	'x'	 & 	'x' \\
 \hfill \nmllink{f\_dardg1dt}{f_dardg1dt}	 & 	'x'	 & 	'x'	 & 	'x'	 & 	'x' \\
 \hfill \nmllink{f\_dardg1ndt}{f_dardg1ndt}	 & 	'x'	 & 	'x'	 & 	'x'	 & 	'x' \\
 \hfill \nmllink{f\_dardg2dt}{f_dardg2dt}	 & 	'x'	 & 	'x'	 & 	'x'	 & 	'x' \\
 \hfill \nmllink{f\_dardg2ndt}{f_dardg2ndt}	 & 	'x'	 & 	'x'	 & 	'x'	 & 	'x' \\
 \hfill \nmllink{f\_dvirdgdt}{f_dvirdgdt}	 & 	'x'	 & 	'x'	 & 	'x'	 & 	'x' \\
 \hfill \nmllink{f\_dvirdgndt}{f_dvirdgndt}	 & 	'x'	 & 	'x'	 & 	'x'	 & 	'x' \\
 \hfill \nmllink{f\_krdgn}{f_krdgn}	 & 	'x'	 & 	'x'	 & 	'x'	 & 	'x' \\
 \hfill \nmldiffer{\nmllink{f\_opening}{f_opening}}	 & 	'x'	 & 	'x'	 & 	'x'	 & 	'm' \\
 \hfill \nmldiffer{\nmllink{f\_vlvl}{f_vlvl}}	 & 	'm'	 & 	'm'	 & 	'm'	 & 	'x' \\
 \hfill \nmllink{f\_vraftn}{f_vraftn}	 & 	'x'	 & 	'x'	 & 	'x'	 & 	'x' \\
 \hfill \nmldiffer{\nmllink{f\_vrdg}{f_vrdg}}	 & 	'm'	 & 	'm'	 & 	'm'	 & 	'x' \\
 \hfill \nmllink{f\_vrdgn}{f_vrdgn}	 & 	'x'	 & 	'x'	 & 	'x'	 & 	'x' \\
 \hfill \nmllink{f\_vredistn}{f_vredistn}	 & 	'x'	 & 	'x'	 & 	'x'	 & 	'x' \\
\hline
\&\nmllink{icefields\_nml}{icefields_nml} \hfill \nmldiffer{\nmllink{f\_aice}{f_aice}}	 & 	'm'	 & 	'm'	 & 	'm'	 & 	'md' \\
 \hfill \nmldiffer{\nmllink{f\_aicen}{f_aicen}}	 & 	'm'	 & 	'm'	 & 	'm'	 & 	'd' \\
 \hfill \nmllink{f\_aisnap}{f_aisnap}	 & 	'x'	 & 	'x'	 & 	'x'	 & 	'x' \\
 \hfill \nmldiffer{\nmllink{f\_albice}{f_albice}}	 & 	'm'	 & 	'm'	 & 	'm'	 & 	'x' \\
 \hfill \nmllink{f\_albpnd}{f_albpnd}	 & 	'x'	 & 	'x'	 & 	'x'	 & 	'x' \\
 \hfill \nmldiffer{\nmllink{f\_albsni}{f_albsni}}	 & 	'm'	 & 	'm'	 & 	'm'	 & 	'x' \\
 \hfill \nmldiffer{\nmllink{f\_albsno}{f_albsno}}	 & 	'm'	 & 	'm'	 & 	'm'	 & 	'x' \\
 \hfill \nmllink{f\_alidr}{f_alidr}	 & 	'x'	 & 	'x'	 & 	'x'	 & 	'x' \\
 \hfill \nmllink{f\_alvdr}{f_alvdr}	 & 	'x'	 & 	'x'	 & 	'x'	 & 	'x' \\
 \hfill \nmllink{f\_angle}{f_angle}	 & 	True	 & 	True	 & 	True	 & 	True \\
 \hfill \nmllink{f\_anglet}{f_anglet}	 & 	True	 & 	True	 & 	True	 & 	True \\
 \hfill \nmllink{f\_bounds}{f_bounds}	 & 	False	 & 	False	 & 	False	 & 	False \\
 \hfill \nmllink{f\_congel}{f_congel}	 & 	'm'	 & 	'm'	 & 	'm'	 & 	'm' \\
 \hfill \nmllink{f\_coszen}{f_coszen}	 & 	'x'	 & 	'x'	 & 	'x'	 & 	'x' \\
 \hfill \nmldiffer{\nmllink{f\_daidtd}{f_daidtd}}	 & 	'm'	 & 	'm'	 & 	'm'	 & 	'x' \\
 \hfill \nmldiffer{\nmllink{f\_daidtt}{f_daidtt}}	 & 	'm'	 & 	'm'	 & 	'm'	 & 	'x' \\
 \hfill \nmldiffer{\nmllink{f\_divu}{f_divu}}	 & 	'm'	 & 	'm'	 & 	'm'	 & 	'md' \\
 \hfill \nmllink{f\_dsnow}{f_dsnow}	 & 	'x'	 & 	'x'	 & 	'x'	 & 	'x' \\
 \hfill \nmldiffer{\nmllink{f\_dvidtd}{f_dvidtd}}	 & 	'm'	 & 	'm'	 & 	'm'	 & 	'x' \\
 \hfill \nmldiffer{\nmllink{f\_dvidtt}{f_dvidtt}}	 & 	'm'	 & 	'm'	 & 	'm'	 & 	'x' \\
 \hfill \nmllink{f\_dxt}{f_dxt}	 & 	True	 & 	True	 & 	True	 & 	True \\
 \hfill \nmllink{f\_dxu}{f_dxu}	 & 	True	 & 	True	 & 	True	 & 	True \\
 \hfill \nmllink{f\_dyt}{f_dyt}	 & 	True	 & 	True	 & 	True	 & 	True \\
 \hfill \nmllink{f\_dyu}{f_dyu}	 & 	True	 & 	True	 & 	True	 & 	True \\
 \hfill \nmllink{f\_evap}{f_evap}	 & 	'x'	 & 	'x'	 & 	'x'	 & 	'x' \\
 \hfill \nmldiffer{\nmllink{f\_evap\_ai}{f_evap_ai}}	 & 	'm'	 & 	'm'	 & 	'm'	 & 	'x' \\
 \hfill \nmldiffer{\nmllink{f\_fcondtop\_ai}{f_fcondtop_ai}}	 & 	'm'	 & 	'm'	 & 	'm'	 & 	'x' \\
 \hfill \nmldiffer{\nmllink{f\_fcondtopn\_ai}{f_fcondtopn_ai}}	 & 	'm'	 & 	'm'	 & 	'm'	 & 	'x' \\
 \hfill \nmllink{f\_fhocn}{f_fhocn}	 & 	'x'	 & 	'x'	 & 	'x'	 & 	'x' \\
 \hfill \nmldiffer{\nmllink{f\_fhocn\_ai}{f_fhocn_ai}}	 & 	'm'	 & 	'm'	 & 	'm'	 & 	'x' \\
 \hfill \nmllink{f\_flat}{f_flat}	 & 	'x'	 & 	'x'	 & 	'x'	 & 	'x' \\
 \hfill \nmldiffer{\nmllink{f\_flat\_ai}{f_flat_ai}}	 & 	'm'	 & 	'm'	 & 	'm'	 & 	'x' \\
 \hfill \nmllink{f\_flatn\_ai}{f_flatn_ai}	 & 	'm'	 & 	'm'	 & 	'm'	 & 	'm' \\
 \hfill \nmldiffer{\nmllink{f\_flwdn}{f_flwdn}}	 & 	'm'	 & 	'm'	 & 	'm'	 & 	'x' \\
 \hfill \nmllink{f\_flwup}{f_flwup}	 & 	'x'	 & 	'x'	 & 	'x'	 & 	'x' \\
 \hfill \nmldiffer{\nmllink{f\_flwup\_ai}{f_flwup_ai}}	 & 	'm'	 & 	'm'	 & 	'm'	 & 	'x' \\
 \hfill \nmldiffer{\nmllink{f\_fmeltt\_ai}{f_fmeltt_ai}}	 & 	'x'	 & 	'x'	 & 	'x'	 & 	'm' \\
 \hfill \nmldiffer{\nmllink{f\_fmelttn\_ai}{f_fmelttn_ai}}	 & 	'm'	 & 	'm'	 & 	'm'	 & 	'x' \\
 \hfill \nmllink{f\_frazil}{f_frazil}	 & 	'm'	 & 	'm'	 & 	'm'	 & 	'm' \\
 \hfill \nmllink{f\_fresh}{f_fresh}	 & 	'x'	 & 	'x'	 & 	'x'	 & 	'x' \\
 \hfill \nmldiffer{\nmllink{f\_fresh\_ai}{f_fresh_ai}}	 & 	'm'	 & 	'm'	 & 	'm'	 & 	'x' \\
 \hfill \nmllink{f\_frz\_onset}{f_frz_onset}	 & 	'm'	 & 	'm'	 & 	'm'	 & 	'm' \\
 \hfill \nmldiffer{\nmllink{f\_frzmlt}{f_frzmlt}}	 & 	'm'	 & 	'm'	 & 	'm'	 & 	'x' \\
 \hfill \nmldiffer{\nmllink{f\_fsalt}{f_fsalt}}	 & 	'x'	 & 	'x'	 & 	'x'	 & 	'd' \\
 \hfill \nmldiffer{\nmllink{f\_fsalt\_ai}{f_fsalt_ai}}	 & 	'm'	 & 	'm'	 & 	'm'	 & 	'd' \\
 \hfill \nmllink{f\_fsens}{f_fsens}	 & 	'x'	 & 	'x'	 & 	'x'	 & 	'x' \\
 \hfill \nmldiffer{\nmllink{f\_fsens\_ai}{f_fsens_ai}}	 & 	'm'	 & 	'm'	 & 	'm'	 & 	'x' \\
 \hfill \nmllink{f\_fsurf\_ai}{f_fsurf_ai}	 & 	'x'	 & 	'x'	 & 	'x'	 & 	'x' \\
 \hfill \nmldiffer{\nmllink{f\_fsurfn\_ai}{f_fsurfn_ai}}	 & 	'm'	 & 	'm'	 & 	'm'	 & 	'x' \\
 \hfill \nmllink{f\_fswabs}{f_fswabs}	 & 	'x'	 & 	'x'	 & 	'x'	 & 	'x' \\
 \hfill \nmldiffer{\nmllink{f\_fswabs\_ai}{f_fswabs_ai}}	 & 	'm'	 & 	'm'	 & 	'm'	 & 	'x' \\
 \hfill \nmldiffer{\nmllink{f\_fswdn}{f_fswdn}}	 & 	'm'	 & 	'm'	 & 	'm'	 & 	'x' \\
 \hfill \nmldiffer{\nmllink{f\_fswfac}{f_fswfac}}	 & 	'm'	 & 	'm'	 & 	'm'	 & 	'x' \\
 \hfill \nmllink{f\_fswthru}{f_fswthru}	 & 	'x'	 & 	'x'	 & 	'x'	 & 	'x' \\
 \hfill \nmldiffer{\nmllink{f\_fswthru\_ai}{f_fswthru_ai}}	 & 	'm'	 & 	'm'	 & 	'm'	 & 	'x' \\
 \hfill \nmllink{f\_fy}{f_fy}	 & 	'x'	 & 	'x'	 & 	'x'	 & 	'x' \\
 \hfill \nmldiffer{\nmllink{f\_hi}{f_hi}}	 & 	'm'	 & 	'm'	 & 	'm'	 & 	'md' \\
 \hfill \nmllink{f\_hisnap}{f_hisnap}	 & 	'x'	 & 	'x'	 & 	'x'	 & 	'x' \\
 \hfill \nmllink{f\_hs}{f_hs}	 & 	'm'	 & 	'm'	 & 	'm'	 & 	'm' \\
 \hfill \nmllink{f\_hte}{f_hte}	 & 	True	 & 	True	 & 	True	 & 	True \\
 \hfill \nmllink{f\_htn}{f_htn}	 & 	True	 & 	True	 & 	True	 & 	True \\
 \hfill \nmllink{f\_iage}{f_iage}	 & 	'm'	 & 	'm'	 & 	'm'	 & 	'm' \\
 \hfill \nmldiffer{\nmllink{f\_icepresent}{f_icepresent}}	 & 	'm'	 & 	'm'	 & 	'm'	 & 	'x' \\
 \hfill \nmldiffer{\nmllink{f\_meltb}{f_meltb}}	 & 	'm'	 & 	'm'	 & 	'm'	 & 	'x' \\
 \hfill \nmldiffer{\nmllink{f\_meltl}{f_meltl}}	 & 	'm'	 & 	'm'	 & 	'm'	 & 	'x' \\
 \hfill \nmldiffer{\nmllink{f\_melts}{f_melts}}	 & 	'm'	 & 	'm'	 & 	'm'	 & 	'x' \\
 \hfill \nmldiffer{\nmllink{f\_meltt}{f_meltt}}	 & 	'm'	 & 	'm'	 & 	'm'	 & 	'x' \\
 \hfill \nmllink{f\_mlt\_onset}{f_mlt_onset}	 & 	'm'	 & 	'm'	 & 	'm'	 & 	'm' \\
 \hfill \nmllink{f\_ncat}{f_ncat}	 & 	True	 & 	True	 & 	True	 & 	True \\
 \hfill \nmllink{f\_qref}{f_qref}	 & 	'x'	 & 	'x'	 & 	'x'	 & 	'x' \\
 \hfill \nmllink{f\_rain}{f_rain}	 & 	'x'	 & 	'x'	 & 	'x'	 & 	'x' \\
 \hfill \nmldiffer{\nmllink{f\_rain\_ai}{f_rain_ai}}	 & 	'm'	 & 	'm'	 & 	'm'	 & 	'x' \\
 \hfill \nmldiffer{\nmllink{f\_shear}{f_shear}}	 & 	'm'	 & 	'm'	 & 	'm'	 & 	'md' \\
 \hfill \nmldiffer{\nmllink{f\_sice}{f_sice}}	 & 	'm'	 & 	'm'	 & 	'm'	 & 	'x' \\
 \hfill \nmldiffer{\nmllink{f\_sig1}{f_sig1}}	 & 	'x'	 & 	'x'	 & 	'x'	 & 	'md' \\
 \hfill \nmldiffer{\nmllink{f\_sig2}{f_sig2}}	 & 	'x'	 & 	'x'	 & 	'x'	 & 	'md' \\
 \hfill \nmllink{f\_sinz}{f_sinz}	 & 	'x'	 & 	'x'	 & 	'x'	 & 	'x' \\
 \hfill \nmldiffer{\nmllink{f\_snoice}{f_snoice}}	 & 	'm'	 & 	'm'	 & 	'm'	 & 	'x' \\
 \hfill \nmllink{f\_snow}{f_snow}	 & 	'x'	 & 	'x'	 & 	'x'	 & 	'x' \\
 \hfill \nmldiffer{\nmllink{f\_snow\_ai}{f_snow_ai}}	 & 	'm'	 & 	'm'	 & 	'm'	 & 	'x' \\
 \hfill \nmldiffer{\nmllink{f\_sss}{f_sss}}	 & 	'm'	 & 	'm'	 & 	'm'	 & 	'd' \\
 \hfill \nmldiffer{\nmllink{f\_sst}{f_sst}}	 & 	'm'	 & 	'm'	 & 	'm'	 & 	'd' \\
 \hfill \nmldiffer{\nmllink{f\_strairx}{f_strairx}}	 & 	'm'	 & 	'm'	 & 	'm'	 & 	'md' \\
 \hfill \nmldiffer{\nmllink{f\_strairy}{f_strairy}}	 & 	'm'	 & 	'm'	 & 	'm'	 & 	'md' \\
 \hfill \nmldiffer{\nmllink{f\_strcorx}{f_strcorx}}	 & 	'm'	 & 	'm'	 & 	'm'	 & 	'x' \\
 \hfill \nmldiffer{\nmllink{f\_strcory}{f_strcory}}	 & 	'm'	 & 	'm'	 & 	'm'	 & 	'x' \\
 \hfill \nmllink{f\_strength}{f_strength}	 & 	'm'	 & 	'm'	 & 	'm'	 & 	'm' \\
 \hfill \nmldiffer{\nmllink{f\_strintx}{f_strintx}}	 & 	'm'	 & 	'm'	 & 	'm'	 & 	'x' \\
 \hfill \nmldiffer{\nmllink{f\_strinty}{f_strinty}}	 & 	'm'	 & 	'm'	 & 	'm'	 & 	'x' \\
 \hfill \nmldiffer{\nmllink{f\_strocnx}{f_strocnx}}	 & 	'm'	 & 	'm'	 & 	'm'	 & 	'x' \\
 \hfill \nmldiffer{\nmllink{f\_strocny}{f_strocny}}	 & 	'm'	 & 	'm'	 & 	'm'	 & 	'x' \\
 \hfill \nmldiffer{\nmllink{f\_strtltx}{f_strtltx}}	 & 	'm'	 & 	'm'	 & 	'm'	 & 	'x' \\
 \hfill \nmldiffer{\nmllink{f\_strtlty}{f_strtlty}}	 & 	'm'	 & 	'm'	 & 	'm'	 & 	'x' \\
 \hfill \nmldiffer{\nmllink{f\_tair}{f_tair}}	 & 	'm'	 & 	'm'	 & 	'm'	 & 	'x' \\
 \hfill \nmllink{f\_tarea}{f_tarea}	 & 	True	 & 	True	 & 	True	 & 	True \\
 \hfill \nmllink{f\_tinz}{f_tinz}	 & 	'x'	 & 	'x'	 & 	'x'	 & 	'x' \\
 \hfill \nmllink{f\_tmask}{f_tmask}	 & 	True	 & 	True	 & 	True	 & 	True \\
 \hfill \nmllink{f\_tref}{f_tref}	 & 	'x'	 & 	'x'	 & 	'x'	 & 	'x' \\
 \hfill \nmldiffer{\nmllink{f\_trsig}{f_trsig}}	 & 	'm'	 & 	'm'	 & 	'm'	 & 	'x' \\
 \hfill \nmllink{f\_tsfc}{f_tsfc}	 & 	'm'	 & 	'm'	 & 	'm'	 & 	'm' \\
 \hfill \nmllink{f\_tsnz}{f_tsnz}	 & 	'x'	 & 	'x'	 & 	'x'	 & 	'x' \\
 \hfill \nmllink{f\_uarea}{f_uarea}	 & 	True	 & 	True	 & 	True	 & 	True \\
 \hfill \nmldiffer{\nmllink{f\_uocn}{f_uocn}}	 & 	'm'	 & 	'm'	 & 	'm'	 & 	'd' \\
 \hfill \nmldiffer{\nmllink{f\_uvel}{f_uvel}}	 & 	'm'	 & 	'm'	 & 	'm'	 & 	'md' \\
 \hfill \nmllink{f\_vgrdb}{f_vgrdb}	 & 	False	 & 	False	 & 	False	 & 	False \\
 \hfill \nmllink{f\_vgrdi}{f_vgrdi}	 & 	False	 & 	False	 & 	False	 & 	False \\
 \hfill \nmllink{f\_vgrds}{f_vgrds}	 & 	False	 & 	False	 & 	False	 & 	False \\
 \hfill \nmllink{f\_vicen}{f_vicen}	 & 	'm'	 & 	'm'	 & 	'm'	 & 	'm' \\
 \hfill \nmldiffer{\nmllink{f\_vocn}{f_vocn}}	 & 	'm'	 & 	'm'	 & 	'm'	 & 	'd' \\
 \hfill \nmldiffer{\nmllink{f\_vvel}{f_vvel}}	 & 	'm'	 & 	'm'	 & 	'm'	 & 	'md' \\
\hline
\&\nmllink{icefields\_pond\_nml}{icefields_pond_nml} \hfill \nmldiffer{\nmllink{f\_apeff}{f_apeff}}	 & 	'm'	 & 	'm'	 & 	'm'	 & 	'x' \\
 \hfill \nmldiffer{\nmllink{f\_apeff\_ai}{f_apeff_ai}}	 & 	'm'	 & 	'm'	 & 	'm'	 & 	'x' \\
 \hfill \nmllink{f\_apeffn}{f_apeffn}	 & 	'x'	 & 	'x'	 & 	'x'	 & 	'x' \\
 \hfill \nmldiffer{\nmllink{f\_apond}{f_apond}}	 & 	'm'	 & 	'm'	 & 	'm'	 & 	'x' \\
 \hfill \nmldiffer{\nmllink{f\_apond\_ai}{f_apond_ai}}	 & 	'm'	 & 	'm'	 & 	'm'	 & 	'x' \\
 \hfill \nmllink{f\_apondn}{f_apondn}	 & 	'x'	 & 	'x'	 & 	'x'	 & 	'x' \\
 \hfill \nmldiffer{\nmllink{f\_hpond}{f_hpond}}	 & 	'm'	 & 	'm'	 & 	'm'	 & 	'x' \\
 \hfill \nmldiffer{\nmllink{f\_hpond\_ai}{f_hpond_ai}}	 & 	'm'	 & 	'm'	 & 	'm'	 & 	'x' \\
 \hfill \nmllink{f\_hpondn}{f_hpondn}	 & 	'x'	 & 	'x'	 & 	'x'	 & 	'x' \\
 \hfill \nmldiffer{\nmllink{f\_ipond}{f_ipond}}	 & 	'm'	 & 	'm'	 & 	'm'	 & 	'x' \\
 \hfill \nmldiffer{\nmllink{f\_ipond\_ai}{f_ipond_ai}}	 & 	'm'	 & 	'm'	 & 	'm'	 & 	'x' \\
\hline
\&\nmllink{ponds\_nml}{ponds_nml} \hfill \nmllink{dpscale}{dpscale}	 & 	\num*{0.001}{}	 & 	\num*{0.001}{}	 & 	\num*{0.001}{}	 & 	\num*{0.001}{} \\
 \hfill \nmllink{frzpnd}{frzpnd}	 & 	'hlid'	 & 	'hlid'	 & 	'hlid'	 & 	'hlid' \\
 \hfill \nmllink{hp1}{hp1}	 & 	\num*{0.01}{}	 & 	\num*{0.01}{}	 & 	\num*{0.01}{}	 & 	\num*{0.01}{} \\
 \hfill \nmllink{hs0}{hs0}	 & 	\num*{0.0}{}	 & 	\num*{0.0}{}	 & 	\num*{0.0}{}	 & 	\num*{0.0}{} \\
 \hfill \nmllink{hs1}{hs1}	 & 	\num*{0.03}{}	 & 	\num*{0.03}{}	 & 	\num*{0.03}{}	 & 	\num*{0.03}{} \\
 \hfill \nmllink{pndaspect}{pndaspect}	 & 	\num*{0.8}{}	 & 	\num*{0.8}{}	 & 	\num*{0.8}{}	 & 	\num*{0.8}{} \\
 \hfill \nmllink{rfracmax}{rfracmax}	 & 	\num*{1.0}{}	 & 	\num*{1.0}{}	 & 	\num*{1.0}{}	 & 	\num*{1.0}{} \\
 \hfill \nmllink{rfracmin}{rfracmin}	 & 	\num*{0.15}{}	 & 	\num*{0.15}{}	 & 	\num*{0.15}{}	 & 	\num*{0.15}{} \\
\hline
\&\nmllink{setup\_nml}{setup_nml} \hfill \nmllink{days\_per\_year}{days_per_year}	 & 	365	 & 	365	 & 	365	 & 	365 \\
 \hfill \nmllink{dbug}{dbug}	 & 	False	 & 	False	 & 	False	 & 	False \\
 \hfill \nmllink{diag\_file}{diag_file}	 & 	'ice\_diag.d'	 & 	'ice\_diag.d'	 & 	'ice\_diag.d'	 & 	'ice\_diag.d' \\
 \hfill \nmllink{diag\_type}{diag_type}	 & 	'file'	 & 	'file'	 & 	'file'	 & 	'file' \\
 \hfill \nmldiffer{\nmllink{diagfreq}{diagfreq}}	 & 	24	 & 	960	 & 	960	 & 	960 \\
 \hfill \nmldiffer{\nmllink{dt}{dt}}	 & 	3600	 & 	1800	 & 	1800	 & 	300 \\
 \hfill \nmllink{dump\_last}{dump_last}	 & 	True	 & 	True	 & 	True	 & 	True \\
 \hfill \nmldiffer{\nmllink{dumpfreq}{dumpfreq}}	 & 	'y'	 & 	'y'	 & 	'y'	 & 	'm' \\
 \hfill \nmldiffer{\nmllink{dumpfreq\_n}{dumpfreq_n}}	 & 	1	 & 	1	 & 	1	 & 	3 \\
 \hfill \nmllink{hist\_avg}{hist_avg}	 & 	True	 & 	True	 & 	True	 & 	True \\
 \hfill \nmllink{histfreq}{histfreq}	 & 	'd', 'm', 'x', 'x', 'x'	 & 	'd', 'm', 'x', 'x', 'x'	 & 	'd', 'm', 'x', 'x', 'x'	 & 	'd', 'm', 'x', 'x', 'x' \\
 \hfill \nmllink{histfreq\_n}{histfreq_n}	 & 	1, 1, 1, 1, 1	 & 	1, 1, 1, 1, 1	 & 	1, 1, 1, 1, 1	 & 	1, 1, 1, 1, 1 \\
 \hfill \nmllink{history\_dir}{history_dir}	 & 	'.\slash OUTPUT\slash '	 & 	'.\slash OUTPUT\slash '	 & 	'.\slash OUTPUT\slash '	 & 	'.\slash OUTPUT\slash ' \\
 \hfill \nmllink{history\_file}{history_file}	 & 	'iceh'	 & 	'iceh'	 & 	'iceh'	 & 	'iceh' \\
 \hfill \nmllink{ice\_ic}{ice_ic}	 & 	'default'	 & 	'default'	 & 	'default'	 & 	'default' \\
 \hfill \nmllink{incond\_dir}{incond_dir}	 & 	'.\slash OUTPUT\slash '	 & 	'.\slash OUTPUT\slash '	 & 	'.\slash OUTPUT\slash '	 & 	'.\slash OUTPUT\slash ' \\
 \hfill \nmllink{incond\_file}{incond_file}	 & 	'iceh\_ic'	 & 	'iceh\_ic'	 & 	'iceh\_ic'	 & 	'iceh\_ic' \\
 \hfill \nmldiffer{\nmllink{istep0}{istep0}}	 & 	2067360	 & 	330336	 & 	167400	 & 	3906432 \\
 \hfill \nmldiffer{\nmllink{latpnt}{latpnt}}	 & 	\num*{90.0}{}, \num*{-65.0}{}	 & 	\num*{90.0}{}, \num*{-65.0}{}	 & 	\num*{90.0}{}, \num*{-65.0}{}	 & 	\num*{66.75}{}, \num*{68.0}{} \\
 \hfill \nmldiffer{\nmllink{lcdf64}{lcdf64}}	 & 	False	 & 	True	 & 	True	 & 	True \\
 \hfill \nmldiffer{\nmllink{lonpnt}{lonpnt}}	 & 	\num*{0.0}{}, \num*{-45.0}{}	 & 	\num*{0.0}{}, \num*{-45.0}{}	 & 	\num*{0.0}{}, \num*{-45.0}{}	 & 	\num*{72.5}{}, \num*{74.0}{} \\
 \hfill \nmldiffer{\nmllink{ndtd}{ndtd}}	 & 	1	 & 	1	 & 	1	 & 	3 \\
 \hfill \nmldiffer{\nmllink{npt}{npt}}	 & 	35040	 & 	2232	 & 	2232	 & 	6480 \\
 \hfill \nmllink{pointer\_file}{pointer_file}	 & 	'.\slash RESTART\slash ice.restart\_file'	 & 	'.\slash RESTART\slash ice.restart\_file'	 & 	'.\slash RESTART\slash ice.restart\_file'	 & 	'.\slash RESTART\slash ice.restart\_file' \\
 \hfill \nmllink{print\_global}{print_global}	 & 	False	 & 	False	 & 	False	 & 	False \\
 \hfill \nmllink{print\_points}{print_points}	 & 	False	 & 	False	 & 	False	 & 	False \\
 \hfill \nmllink{restart}{restart}	 & 	True	 & 	True	 & 	True	 & 	True \\
 \hfill \nmllink{restart\_dir}{restart_dir}	 & 	'.\slash RESTART\slash '	 & 	'.\slash RESTART\slash '	 & 	'.\slash RESTART\slash '	 & 	'.\slash RESTART\slash ' \\
 \hfill \nmllink{restart\_ext}{restart_ext}	 & 	False	 & 	False	 & 	False	 & 	False \\
 \hfill \nmllink{restart\_file}{restart_file}	 & 	'iced'	 & 	'iced'	 & 	'iced'	 & 	'iced' \\
 \hfill \nmllink{restart\_format}{restart_format}	 & 	'nc'	 & 	'nc'	 & 	'nc'	 & 	'nc' \\
 \hfill \nmllink{runtype}{runtype}	 & 	'continue'	 & 	'continue'	 & 	'continue'	 & 	'continue' \\
 \hfill \nmllink{use\_leap\_years}{use_leap_years}	 & 	False	 & 	False	 & 	False	 & 	False \\
 \hfill \nmllink{use\_restart\_time}{use_restart_time}	 & 	True	 & 	True	 & 	True	 & 	True \\
 \hfill \nmllink{write\_ic}{write_ic}	 & 	False	 & 	False	 & 	False	 & 	False \\
 \hfill \nmllink{year\_init}{year_init}	 & 	1	 & 	1	 & 	1	 & 	1 \\
\hline
\&\nmllink{shortwave\_nml}{shortwave_nml} \hfill \nmllink{ahmax}{ahmax}	 & 	\num*{0.1}{}	 & 	\num*{0.1}{}	 & 	\num*{0.1}{}	 & 	\num*{0.1}{} \\
 \hfill \nmllink{albedo\_type}{albedo_type}	 & 	'default'	 & 	'default'	 & 	'default'	 & 	'default' \\
 \hfill \nmllink{albicei}{albicei}	 & 	\num*{0.44}{}	 & 	\num*{0.44}{}	 & 	\num*{0.44}{}	 & 	\num*{0.44}{} \\
 \hfill \nmllink{albicev}{albicev}	 & 	\num*{0.86}{}	 & 	\num*{0.86}{}	 & 	\num*{0.86}{}	 & 	\num*{0.86}{} \\
 \hfill \nmllink{albsnowi}{albsnowi}	 & 	\num*{0.7}{}	 & 	\num*{0.7}{}	 & 	\num*{0.7}{}	 & 	\num*{0.7}{} \\
 \hfill \nmllink{albsnowv}{albsnowv}	 & 	\num*{0.98}{}	 & 	\num*{0.98}{}	 & 	\num*{0.98}{}	 & 	\num*{0.98}{} \\
 \hfill \nmllink{dalb\_mlt}{dalb_mlt}	 & 	\num*{-0.02}{}	 & 	\num*{-0.02}{}	 & 	\num*{-0.02}{}	 & 	\num*{-0.02}{} \\
 \hfill \nmllink{dt\_mlt}{dt_mlt}	 & 	\num*{1.0}{}	 & 	\num*{1.0}{}	 & 	\num*{1.0}{}	 & 	\num*{1.0}{} \\
 \hfill \nmllink{r\_ice}{r_ice}	 & 	\num*{0.0}{}	 & 	\num*{0.0}{}	 & 	\num*{0.0}{}	 & 	\num*{0.0}{} \\
 \hfill \nmllink{r\_pnd}{r_pnd}	 & 	\num*{0.0}{}	 & 	\num*{0.0}{}	 & 	\num*{0.0}{}	 & 	\num*{0.0}{} \\
 \hfill \nmllink{r\_snw}{r_snw}	 & 	\num*{0.0}{}	 & 	\num*{0.0}{}	 & 	\num*{0.0}{}	 & 	\num*{0.0}{} \\
 \hfill \nmllink{rsnw\_mlt}{rsnw_mlt}	 & 	\num*{1500.0}{}	 & 	\num*{1500.0}{}	 & 	\num*{1500.0}{}	 & 	\num*{1500.0}{} \\
 \hfill \nmllink{shortwave}{shortwave}	 & 	'default'	 & 	'default'	 & 	'default'	 & 	'default' \\
 \hfill \nmllink{tocnfrz}{tocnfrz}	 & 	\num*{-1.8}{}	 & 	\num*{-1.8}{}	 & 	\num*{-1.8}{}	 & 	\num*{-1.8}{} \\
\hline
\&\nmllink{thermo\_nml}{thermo_nml} \hfill \nmllink{a\_rapid\_mode}{a_rapid_mode}	 & 	\num*{0.0005}{}	 & 	\num*{0.0005}{}	 & 	\num*{0.0005}{}	 & 	\num*{0.0005}{} \\
 \hfill \nmllink{aspect\_rapid\_mode}{aspect_rapid_mode}	 & 	\num*{1.0}{}	 & 	\num*{1.0}{}	 & 	\num*{1.0}{}	 & 	\num*{1.0}{} \\
 \hfill \nmllink{chio}{chio}	 & 	\num*{0.004}{}	 & 	\num*{0.004}{}	 & 	\num*{0.004}{}	 & 	\num*{0.004}{} \\
 \hfill \nmllink{conduct}{conduct}	 & 	'bubbly'	 & 	'bubbly'	 & 	'bubbly'	 & 	'bubbly' \\
 \hfill \nmllink{dsdt\_slow\_mode}{dsdt_slow_mode}	 & 	\num*{-5e-8}{}	 & 	\num*{-5e-8}{}	 & 	\num*{-5e-8}{}	 & 	\num*{-5e-8}{} \\
 \hfill \nmllink{kitd}{kitd}	 & 	1	 & 	1	 & 	1	 & 	1 \\
 \hfill \nmldiffer{\nmllink{ktherm}{ktherm}}	 & 	1	 & 	1	 & 	1	 & 	2 \\
 \hfill \nmllink{phi\_c\_slow\_mode}{phi_c_slow_mode}	 & 	\num*{0.05}{}	 & 	\num*{0.05}{}	 & 	\num*{0.05}{}	 & 	\num*{0.05}{} \\
 \hfill \nmllink{phi\_i\_mushy}{phi_i_mushy}	 & 	\num*{0.85}{}	 & 	\num*{0.85}{}	 & 	\num*{0.85}{}	 & 	\num*{0.85}{} \\
 \hfill \nmllink{rac\_rapid\_mode}{rac_rapid_mode}	 & 	\num*{10.0}{}	 & 	\num*{10.0}{}	 & 	\num*{10.0}{}	 & 	\num*{10.0}{} \\
\hline
\&\nmllink{tracer\_nml}{tracer_nml} \hfill \nmllink{restart\_aero}{restart_aero}	 & 	False	 & 	False	 & 	False	 & 	False \\
 \hfill \nmllink{restart\_age}{restart_age}	 & 	False	 & 	False	 & 	False	 & 	False \\
 \hfill \nmllink{restart\_fy}{restart_fy}	 & 	False	 & 	False	 & 	False	 & 	False \\
 \hfill \nmllink{restart\_lvl}{restart_lvl}	 & 	False	 & 	False	 & 	False	 & 	False \\
 \hfill \nmllink{restart\_pond\_cesm}{restart_pond_cesm}	 & 	False	 & 	False	 & 	False	 & 	False \\
 \hfill \nmllink{restart\_pond\_lvl}{restart_pond_lvl}	 & 	False	 & 	False	 & 	False	 & 	False \\
 \hfill \nmllink{restart\_pond\_topo}{restart_pond_topo}	 & 	False	 & 	False	 & 	False	 & 	False \\
 \hfill \nmllink{tr\_aero}{tr_aero}	 & 	False	 & 	False	 & 	False	 & 	False \\
 \hfill \nmllink{tr\_fy}{tr_fy}	 & 	False	 & 	False	 & 	False	 & 	False \\
 \hfill \nmllink{tr\_iage}{tr_iage}	 & 	False	 & 	False	 & 	False	 & 	False \\
 \hfill \nmllink{tr\_lvl}{tr_lvl}	 & 	False	 & 	False	 & 	False	 & 	False \\
 \hfill \nmllink{tr\_pond\_cesm}{tr_pond_cesm}	 & 	False	 & 	False	 & 	False	 & 	False \\
 \hfill \nmllink{tr\_pond\_lvl}{tr_pond_lvl}	 & 	False	 & 	False	 & 	False	 & 	False \\
 \hfill \nmllink{tr\_pond\_topo}{tr_pond_topo}	 & 	False	 & 	False	 & 	False	 & 	False \\
\hline
\&\nmllink{zbgc\_nml}{zbgc_nml} \hfill \nmllink{bgc\_data\_dir}{bgc_data_dir}	 & 	'unknown\_bgc\_data\_dir'	 & 	'unknown\_bgc\_data\_dir'	 & 	'unknown\_bgc\_data\_dir'	 & 	'unknown\_bgc\_data\_dir' \\
 \hfill \nmllink{bgc\_flux\_type}{bgc_flux_type}	 & 	'Jin2006'	 & 	'Jin2006'	 & 	'Jin2006'	 & 	'Jin2006' \\
 \hfill \nmllink{nit\_data\_type}{nit_data_type}	 & 	'default'	 & 	'default'	 & 	'default'	 & 	'default' \\
 \hfill \nmllink{phi\_snow}{phi_snow}	 & 	\num*{0.5}{}	 & 	\num*{0.5}{}	 & 	\num*{0.5}{}	 & 	\num*{0.5}{} \\
 \hfill \nmllink{restart\_bgc}{restart_bgc}	 & 	False	 & 	False	 & 	False	 & 	False \\
 \hfill \nmllink{restart\_hbrine}{restart_hbrine}	 & 	False	 & 	False	 & 	False	 & 	False \\
 \hfill \nmllink{restore\_bgc}{restore_bgc}	 & 	False	 & 	False	 & 	False	 & 	False \\
 \hfill \nmllink{sil\_data\_type}{sil_data_type}	 & 	'default'	 & 	'default'	 & 	'default'	 & 	'default' \\
 \hfill \nmllink{skl\_bgc}{skl_bgc}	 & 	False	 & 	False	 & 	False	 & 	False \\
 \hfill \nmllink{tr\_bgc\_am\_sk}{tr_bgc_am_sk}	 & 	False	 & 	False	 & 	False	 & 	False \\
 \hfill \nmllink{tr\_bgc\_c\_sk}{tr_bgc_c_sk}	 & 	False	 & 	False	 & 	False	 & 	False \\
 \hfill \nmllink{tr\_bgc\_chl\_sk}{tr_bgc_chl_sk}	 & 	False	 & 	False	 & 	False	 & 	False \\
 \hfill \nmllink{tr\_bgc\_dms\_sk}{tr_bgc_dms_sk}	 & 	False	 & 	False	 & 	False	 & 	False \\
 \hfill \nmllink{tr\_bgc\_dmspd\_sk}{tr_bgc_dmspd_sk}	 & 	False	 & 	False	 & 	False	 & 	False \\
 \hfill \nmllink{tr\_bgc\_dmspp\_sk}{tr_bgc_dmspp_sk}	 & 	False	 & 	False	 & 	False	 & 	False \\
 \hfill \nmllink{tr\_bgc\_sil\_sk}{tr_bgc_sil_sk}	 & 	False	 & 	False	 & 	False	 & 	False \\
 \hfill \nmllink{tr\_brine}{tr_brine}	 & 	False	 & 	False	 & 	False	 & 	False \\
\hline
\end{tabularx}
}

\subsection{MATM namelist `input\_atm.nml'}
\renewcommand{\link}[2]{\href{https://github.com/OceansAus/matm/search?q=#2}{#1}} % link to documentation (requires hyperref package)
{\tiny 
% File auto-generated by nmltab.py <https://github.com/aekiss/nmltab>
% Requires ltablex, array and sistyle packages
% Also need to define 'nmldiffer' and 'nmllink' commands, e.g.
% \newcommand{\nmldiffer}[1]{#1} % no special display of differing variables
% \newcommand{\nmldiffer}[1]{\textbf{#1}} % bold display of differing variables
% \definecolor{hilite}{cmyk}{0, 0, 0.9, 0}\newcommand{\nmldiffer}[1]{\colorbox{hilite}{#1}}\setlength{\fboxsep}{0pt} % colour highlight of differing variables (requires color package)
% \newcommand{\nmllink}[2]{#1} % don't link variables
% \newcommand{\nmllink}[2]{\href{https://github.com/mom-ocean/MOM5/search?q=#2}{#1}} % link variables to documentation (requires hyperref package)
% and also the length 'nmllen', e.g.
% \newlength{\nmllen}\setlength{\nmllen}{12ex}

\newcolumntype{R}{>{\raggedleft\arraybackslash}p{\nmllen}}
\begin{tabularx}{\linewidth}{XRRR}
\hline
\textbf{Group\hfill Variable}	 & 	\textbf{\slash short\slash v45\slash amh157\slash access-om2\slash control\slash 1deg\_jra55\_ryf\slash atmosphere\slash input\_atm.nml}	 & 	\textbf{\slash short\slash v45\slash aek156\slash access-om2\slash control\slash 025deg\_jra55\_ryf\slash atmosphere\slash input\_atm.nml}	 & 	\textbf{\slash short\slash v45\slash amh157\slash access-om2\slash control\slash 01deg\_jra55\_ryf\slash atmosphere\slash input\_atm.nml} \\
\hline\endfirsthead
\hline
\textbf{Group (continued)\hfill Variable}	 & 	\textbf{\slash short\slash v45\slash amh157\slash access-om2\slash control\slash 1deg\_jra55\_ryf\slash atmosphere\slash input\_atm.nml}	 & 	\textbf{\slash short\slash v45\slash aek156\slash access-om2\slash control\slash 025deg\_jra55\_ryf\slash atmosphere\slash input\_atm.nml}	 & 	\textbf{\slash short\slash v45\slash amh157\slash access-om2\slash control\slash 01deg\_jra55\_ryf\slash atmosphere\slash input\_atm.nml} \\
\hline\endhead
coupling \hfill \nmllink{caltype}{caltype}	 & 	0	 & 	0	 & 	0 \\
 \hfill \nmllink{\nmldiffer{chk\_a2i\_fields}}{chk_a2i_fields}	 & 	False	 & 	False	 & 	 \\
 \hfill \nmllink{\nmldiffer{chk\_i2a\_fields}}{chk_i2a_fields}	 & 	False	 & 	False	 & 	 \\
 \hfill \nmllink{dataset}{dataset}	 & 	'jra55'	 & 	'jra55'	 & 	'jra55' \\
 \hfill \nmllink{days\_per\_year}{days_per_year}	 & 	365	 & 	365	 & 	365 \\
 \hfill \nmllink{\nmldiffer{debug\_output}}{debug_output}	 & 	False	 & 		 & 	 \\
 \hfill \nmllink{\nmldiffer{dt\_atm}}{dt_atm}	 & 	3600	 & 	1200	 & 	400 \\
 \hfill \nmllink{dt\_cpl}{dt_cpl}	 & 	10800	 & 	10800	 & 	10800 \\
 \hfill \nmllink{inidate}{inidate}	 & 	10101	 & 	10101	 & 	10101 \\
 \hfill \nmllink{init\_date}{init_date}	 & 	10101	 & 	10101	 & 	10101 \\
 \hfill \nmllink{\nmldiffer{runtime}}{runtime}	 & 	126144000	 & 	2678400	 & 	2592000 \\
 \hfill \nmllink{runtype}{runtype}	 & 	'NY'	 & 	'NY'	 & 	'NY' \\
 \hfill \nmllink{truntime0}{truntime0}	 & 	0	 & 	0	 & 	0 \\
\hline
\end{tabularx}
}

\bibliographystyle{ametsoc2014}
\bibliography{ACCESS-OM2-1-025-010deg}

\end{document}  